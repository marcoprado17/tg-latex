A tecnologia blockchain foi pioneira no desenvolvimento de um banco de dados distribuído, imutável, auditável e que resolvia o problema do gasto duplo. Com a crescente evolução da tecnologia blockchain, surgiram os contratos inteligentes, que possibilitaram a criação e execução de códigos em uma rede distribuída, na qual cada nó não precisa confiar um no outro. Apesar de ser uma solução excelente para o armazenamento de transações financeiras, o custo e problemas de performance para armazenar outros tipos de dados na rede blockchain tornam necessário o uso de uma solução híbrida, no qual parte dos dados está no blockchain e outra parte dos dados está em um banco de dados tradicional. Este trabalho apresenta o estudo de caso de um contrato inteligente para seguro automotivos que compara os prós e contras de uma primeira arquitetura, na qual, os dados estão 100\% na rede blockchain e outra segunda arquitetura, na qual, parte dos dados se encontra no blockchain e outra parte dos dados se encontra em uma banco de dados tradicional. Devido a ausência de tecnologias eficientes para a comunicação direta com a rede blockchain, será utilizada, em ambas arquituras, microserviços responsáveis por intermediar o envio dos dados para o blockchain e para o banco de dados tradicional.