Blockchain technology was pionner in the development of a distributed, immutable, auditable database that solved the problem of double spending. With the increasing evolution of blockchain technology, smart contracts have emerged, enabling the creation and execution of codes in a distributed network, in which each node doesn't need to trust each other. Although it is an excellent solution for the storage of financial transactions, the cost and performance issues to store other types of data in the blockchain network make it necessary to use a hybrid solution, in which part of the data is in the blockchain and another part of the data is in a traditional database. This paper presents the case study of a smart contract for automotive insurance that compares the pros and cons of a first architecture, in which the data is 100 \% in the blockchain network and another second architecture, in which part of the data is in the blockchain and another part of the data is in a traditional database. Due to the absence of efficient technologies for direct communication with the blockchain network, will be used, in both architectures, microservices responsible for mediating the sending of data to the blockchain and to the traditional database.