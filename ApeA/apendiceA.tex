% \section{Uma Primeira Seção para o Apêndice}

% A matriz de Dilema Linear $M$ e o vetor de torques inerciais $b$,
% utilizados na simulação são calculados segundo a formulação 
% abaixo:
% \begin{equation}
% M=\left[ \begin{array}{ccc}
% M_{11} & M_{12} & M_{13} \\
% M_{21} & M_{22} & M_{23} \\
% M_{31} & M_{32} & M_{33}
% \end{array} \right]
% \end{equation}

% \begin{figure}[h]
% \centering
% \includegraphics[height=5cm, width=5cm]{ApeA/pragas_ciclo_cupim}
% \caption{Uma figura que está no apêndice}\label{FD}
% \end{figure}

\begin{enumerate}
\item\label{links:express} Express <http://www.express.com>
\item\label{links:scife-ui-service} scife-ui-service <http://www.github.com/marcoprado17/scife-ui-service>
\end{enumerate}